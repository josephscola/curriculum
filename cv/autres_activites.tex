\section{Autres activités}

	\subsection{Cafés pédagogiques}
Je suis à l'initiative des cafés pédagogiques de l'UFR des Sciences de l'UVSQ. J'anime les séances depuis juin 2017. Il s'agit d'un rendez-vous informel d'échanges entre enseignants. Divers sujets sont proposés pour ouvrir la discussion : actualités du monde de l'enseignement, approches pédagogiques, spécificités des disciplines, focus sur une formation, vulgarisation, accueil de nouveaux enseignants, rencontre avec les étudiants, ouverture sur les autres composantes de l'UVSQ, d'autres établissements\dots Tous les enseignants souhaitant discuter de pédagogie autour d'un café y sont conviés chaque semaine.

Les cafés permettent de diffuser les informations concernant les conférences et les formations dans la région et ont également servi de cadre pour de tels évènement. Les discussions permettent d'enrichir sa pratique pédagogique en découvrant celles de ses collègues issus d'autres disciplines. Régulièrement, des projets pédagogiques collaboratifs émergent de ces  discussions et se mettent en place dans un cadre plus formel.
Les cafés pédagogiques sont soutenus par la direction de l'UFR

	\subsection{Correspondant UVSQ du groupe de travail pour les initiatives et l'innovations pédagogiques à Paris-Saclay}


Je participe depuis 2018 au groupe de travail pour les initiatives et l'innovations pédagogiques à Paris-Saclay. L'objet du groupe de travail de contribuer activement aux actions mises en \oe uvre dans la cadre de cette mission. En pratique, il organise des évènements qui rassemblent les enseignants des établissements de Paris-Saclay autour de questions liés à la pédagogie (veille sur les innovations locales, conférences, ateliers et débats sur des questions particulières). Le groupe de travail publie chaque année des appels à projets d'innovations pédagogiques et en évalue les propositions déposées.
Localement, le rôle du correspondant consiste à diffuser les informations d'actualité liées à la pédagogie auprès des collègues, notamment par le moyen des cafés pédagogiques et d'accompagner les porteurs de projets dans leurs démarche de soumission et de gestion administrative des projets acceptés. 