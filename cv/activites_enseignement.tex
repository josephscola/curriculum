\section{Activités d'enseignement}
\cventry
	{Licence 1}
	{Méthodologie pour la physique (Responsabilité de module, CM, TD, TP)}
	{licence math-physique-chimie}
	{(depuis 2018)}{}{}
\cventry
	{}
	{Mécanique générale (Responsabilité de module, CM, TD, TP)}
	{licence math-physique-chimie-informatique}
	{(2010-2020)}{}{}
\cventry
	{}
	{Optique géométrique et électrocinétique (TD, TP)}
	{licence math-physique-chimie-informatique}
	{(2007-2020)}{}{}	
\cventry
	{}
	{Physique pour la PACES (TD)}
	{Première année commune aux études de santé}
	{(2007-2017)}{}{}
\cventry
	{}
	{\'Electronique analogique et numérique (TD)}
	{licence math-physique-chimie-informatique}
	{(2007-2009)}{}{}

\cventry
	{Master 1}
	{Calculs scientifiques (Responsabilité de module, CM, TD, TP)}
	{master Physique, mécanique et science pour l'ingénieur}
	{(2010-2015)}{}{}
	
\cventry%
	{}	%
	{Mesures magnétiques (TP)}
	{master Physique et science pour l'ingénieur}
	{(2010-2012)}{}{}	

\cventry
	{Master 2}
	{Physique et Chimie de la Matière (Responsabilité de module, cours, TD)}
	{master Polymères et Biomatériaux (U. Paris-Saclay)}
	{(depuis 2015)}
	{}{}

\cventry
	{}
	{Physique des semiconducteurs, hétérostructures (TD)}
	{master Matériaux, Technologies et Composants: Photovoltaïque - Voiture Electrique}
	{(2011-2018)}
	{}{}

\iffalse
\cventry
	{}
	{Expériences récentes en Physique quantique (TD)}
	{master Nanosciences (U. Paris-Saclay)}
	{(2013)}
	{}{}
\fi

\cventry
	{}
	{Tuteur de stage}
	{master Métiers de l'Enseignement, de l'\'Education et de la Formation}
	{(depuis 2018)}
	{}{}

\cventry
	{Formation continue}
	{Physique générale (CM, TD)}
	{Diplôme d'accès aux études universitaires, Sciences}
	{(2016-2018)}{}{}


\section{Encadrement d'étudiants}
\cventry
	{Thèse}
	{Co-encadrant}
	{Nanoparticules d'argent pour l'électronique}
	{Yana Veniaminova, 2010-2013}
	{Co-tutelle GEMaC (CNRS-UMR8635) - ILV (CNRS-UMR8180)}{}

\cventry
	{Stages M2}
	{Encadrant}
	{Matériaux pour la récupération d’énergie : propriétés mécaniques de nanofils de ZnO}
	{Idris Aboubakari, 5 mois}	
	{}{}
	
\cventry
	{}
	{Encadrant}
	{Interface Al/ZnO : émergence de propriétés à l'échelle nanométrique}
	{Cheikh Samb, 2019, 4 mois}	
	{}{}
	
\cventry
	{}
	{Encadrant}
	{Mécanisme de conduction dans le métal à fortes corrélations électroniques LaNiO$_3$}
	{Abdelnour Benamar, 2016, 5 mois}	
	{}{}

\cventry
	{}
	{Encadrant}
	{Mesure de résistance à 1000 K}
	{Maya Taoui, 2014, 5 mois}	
	{}{}

\cventry
	{}
	{Encadrant}
	{Nanomatériaux pour l'interconnexion de cellules photovoltaïques à concentration}
	{Nour El Houda Kriden, 2014, 5 mois}
	{}{}
	
\cventry
	{Projet de fin d'étude INSA}
	{Co-encadrant}
	{Simulation numérique d'un antiferromagnétique canté par un modèle macro-spin}
	{Maxime Vallée, 2010, 6 mois}
	{Co-tutelle GEMac - Laboratoire de mathématique de l'INSA de Rouen}{}
	
\cventry
	{Stage M1}
	{Encadrant}
	{Influence des lacunes d'oxygène sur la structure électronique du métal corrélé
	LaNiO$_{3-\delta}$}
	{Mohammed Jamali, 2017, 4 mois}{}{}
	
\cventry
	{Stages L3}
	{Encadrement de 10 binômes en stage d'un mois depuis 2008}
	{}
	{}	
	{}{}

\iffalse
\cventry
	{Stages L3}
	{}
	{"Oxytronique : de l'expérience à la médiation scientifique"}
	{2015, 1 mois}	
	{}{}

\cventry
	{}
	{}
	{"Propriétés magnétiques de nanoparticules magnétiques fonctionnalisées"}
	{2013, 1 mois}	
	{}{}

\cventry
	{}
	{}
	{"Résolution numérique de l'équation de la chaleur"}
	{2011, 1 mois}	
	{}{}
	
\cventry
	{}
	{}
	{"Protocole d'utilisation du Physical Properties Measurement System"}
	{2010, 1 mois}	
	{}{}

\cventry
	{}
	{}
	{"Protocole d'utilisation d'un bâti de pulvérisation cathodique"}
	{2009, 1 mois}	
	{}{}	

\cventry
	{}
	{}
	{"Magnétométrie SQUID"}
	{2009, 1 mois}
	{}{}	

\cventry
	{}
	{}
	{"Conception et réalisation d'amplificateurs analogiques faible bruit"}
	{2008, 1 mois}
	{}{}		
	
%\iffalse	
\section{Contributions au département de physique}
\cventry
	{2012}
	{Médiation scientifique}{}
	{UVSQ}
	{Organisation d'une conférence sur \emph{Le boson de Higgs} par G. Cohen-Tannoudji}
	{}{}{}

\cventry
	{2010-2015}
	{Président de jury}
	{Semestre 2 de la première année de Licence, parcours Physique-Chimie}
	{}
	{}{}	
\fi