Le projet de recherche proposé par M. Joseph SCOLA porte sur la caractérisation microstructurale de nanomatériaux pour l'énergie.
L'équipe d'accueil au GEMaC, Nanostructure Semiconductrice et Propriétés (NSP), est engagée dans la recherche et le développement de nanofils de ZnO pour la récupération d'énergie par couplage électromécanique (flagship NanoVIBES de NanoSaclay).
Les expériences de microstructure par microscopie électronique à transmission prévues dans ce projet seront réalisées sur les équipements du Laboratoire d'Etude de la Microstructure (LEM, UMR 104), en collaboration avec ses membres.
Une étude conjointe entre le GEMaC et le LEM sur des nanofils de ZnS est actuellement en cours et le projet que M. Joseph SCOLA vise à renforcer cette activité.
Cette consolidation de la collaboration entre les deux laboratoires s’inscrit dans le projet de microscopie MOSTRA et plus particulièrement son développement instrumental.